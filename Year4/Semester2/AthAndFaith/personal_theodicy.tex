\documentclass[12pt]{article}
\usepackage{array,tabularx}
\usepackage{graphicx}
\usepackage{float}
\usepackage{amsmath}
\setlength{\parindent}{0pt}

\newenvironment{conditions}
  {\par\vspace{\abovedisplayskip}\noindent
   \tabularx{\columnwidth}{>{$}l<{$}@{}>{${}}c<{{}$}@{} >{\raggedright\arraybackslash}X}}
  {\endtabularx\par\vspace{\belowdisplayskip}}

\title{Personal Theodicy Reflection}
\author{Ben Hammond}
\date{\today}


\begin{document}
	\maketitle
	\newpage

	“They constantly try to escape
	From the darkness outside and within
	By dreaming of systems so perfect that no one will need to be good.
	But the man that is will shadow
	The man that pretends to be.”
	
	― T.S. Eliot

	\paragraph{1. What is the event and how did this event come about?}
	I've been working with depression for a while now. It really became an issue toward the end of my junior year, and was just getting worse for the most part since then. I don't think there was anything in particular which caused this, though the pandemic probably hasn't helped.

	\paragraph{2. What were some of the challenges and how did it impact you?}
	The most obvious challenge was navigating school. It turns out it's difficult to manage a school workload if getting out of bed is a challenge. Toward the end of both my junior year last year, and the end of this semester, my grades started dropping in a sort of snowball effect. If I'm having a bad day, then it's hard to do school work, but then when I don't do my work I feel worse. And so the cycle could repeat, until I was absurdly far behind.

	\paragraph{3. Where are you now in this process (beginning, middle, end)? Explain briefly.}
	I definitely hope I'm toward the end of this, though I'm not sure I can expect it to magically go away. I think I've started to take it a bit more seriously recently, and just focusing on my mental health a bit more seems to have helped, as well as having a very supporting family. I think just finishing school will definitely help (you know, as long as I don't fail...), and I'm looking forward to some plans I have over the summer.

	\paragraph{4. How has it impacted you negatively?}
	Where to begin? For a while, a couple of weeks ago, my GPA was around a 1.9, which i think is a record for me. I'm still digging myself out of that particular hole, and if I do one missing assignment a day, I'll be done just before I graduate. A few weeks ago I also had a close call, which was especially concerning for my family and I.

	\paragraph{5. Has it impacted you in a positive way? Why or why not?}
	The only redeeming consequence of this has been that I'm now a bit more focused on making sure I'm doing okay, but that's a bit of a stretch. I don't think it has been a positive experience overall, and I definitely wouldn't recommend it. I do at least know that I can get through difficult times, though not exactly cleanly.

	\paragraph{6. What can be the deeper meaning here? What is it?}
	I don't think there is a deeper meaning here. Ultimately I think it's just some chemical imbalance in my brain somewhere, and I don't think that has any greater meaning. I think trying to give it some greater purpose just undermines the struggle and pain which I've felt as I'm getting through this.

	\paragraph{7. Who has helped you to get through this period or event? How so? Or were you on your own?}
	My parents have helped a great deal. After a particularly scary incident it's been all hands on deck making sure that I'm doing okay, which I'm very grateful for. My therapist has also been very helpful, and my doing better is in large part due to all of them.
	\paragraph{9. How has this event impacted your relationship or view of God? Explain (if you do not believe in God, then explain how it has impacted you, your position on religion or God at this point in your life).}
	I still don't believe in god, and my previous feelings still hold true for me. I think that there's no way to unify a god's goodness and greatness in the face of all everyone, myself included have gone through. I also don't think that the existence of suffering actually changes anything about the argument for or against the existence of god; they're not mutually exclusive or inclusive. While I do think that religion, and the idea of a god, can be a comfort for many people going through difficult times, I don't personally see the need. For me, the most important aspects of my experiences have come from myself, and my friends and family, not the idea of a god, all-powerful or otherwise.

	\paragraph{10. Now choose a quote (informal annotation) from the philosophy or theme from any of the material we covered in this unit- The Story of Job, Theodicy Defined, The Problem of Evil. What does this quote or theme say to you personally? Agree or disagree and why. What have you learned about yourself and where does it lead from here? How your suffering helped you to grow? And how can it help you to be more compassionate with yourself? Others?}
	"Suffering is simply a part of reality, and if we saw all of reality we would realize that nothing is imperfect" (Spinoza, paraphrased). I agree with part of this. I do think that suffering is simply a part of human existence, and our reality, but I don't think that there is some higher level of understanding which would reveal everything, suffering included, as perfect. I don't think that my experience is indicative of some greater plane of existence, though it may be archetypal for people of my age in modern times. For me, it leads where it always does: forward through the inescapable flow of time. In more concrete terms, I'm planning on going camping with my friends over the summer, traveling to Norway with my mum, and then going to UC Santa Cruz for the next four years.

\end{document}

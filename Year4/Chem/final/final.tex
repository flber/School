\documentclass[12pt]{article}
\usepackage{array,tabularx}
\usepackage{graphicx}
\usepackage{float}
\usepackage{amsmath}
\setlength{\parindent}{0pt}

\newenvironment{conditions}
  {\par\vspace{\abovedisplayskip}\noindent
   \tabularx{\columnwidth}{>{$}l<{$}@{}>{${}}c<{{}$}@{} >{\raggedright\arraybackslash}X}}
  {\endtabularx\par\vspace{\belowdisplayskip}}

\title{Reaction Kinetics Lab Report}
\author{Ben Hammond}
\date{\today}


\begin{document}
	\maketitle
	\newpage

	\section{Purpose}
	The purpose of this lab was to experimentally determine the reaction kinetics for the decomposition of hydrogen peroxide ($\text{H}_2\text{O}_2$) with a catalyst of potassium iodide (KI), including the reaction rate and activation energy.
	
	\section{Procedure}
	A glass cup was first filled with water up to the brim. A 100 m: graduated cylinder was then filled completely with water, and was then inverted into the cup such that no air entered the graduated cylinder. Two glasses of the same height as the first were placed on either side of the water filled one, and two wine glasses were placed on top of the glasses, each one balanced between the water filled glass and one of the two other glasses. The graduated cylinder was then lifted upward and balanced upon the rims of the wine glasses. A long rubber tube with one end inserted into a stopper for a test tube was fed into the water filled graduated cylinder such that the end of tube opposite of the stopper was placed approximately halfway up inside of the graduated cylinder. 6 mL of approximately 3\% $\text{H}_2\text{O}_2$ was measured in a separate graduated cylinder and poured into a test tube. 15 drops of KI were then added to the test tube, and the end of the rubber tubing with the stopper was quickly used to seal the test tube. The volume of $\text{O}_2$ gas collected in the graduated cylinder was measured every thirty seconds until the reaction slowed to an unnoticeable rate.
	
	The setup was then disassembled, cleaned, and reassembled. The second version of the experiment involved holding the reaction at a constant temperature of 50$^\circ$C, which required slight modifications to the experimental setup. The existing setup was moved next to a gas stove top, on top of which was placed a large, flat, metal griddle. This was heated on one end by a low flame until hot. In the middle of the griddle was placed a large metal pot of ice water. At the far end of the griddle, farthest away from the flame, was placed a small pot of water into which the reaction test tube was placed. This allowed the reaction to be kept at a semi-stable temperature, which was kept close to 50$^\circ$C by periodically adding cold water to decrease the temperature or moving it closer to the heat source to increase the temperature. In this way a temperature close to 50$^\circ$C (varying only by a degree or so) was maintained throughout the duration of the reaction. As before, the reaction was started, and the $\text{O}_2$ output was measured throughout the reaction.

	\section{Data}
	\begin{flushleft}
	\begin{tabular}{ l|l|l }
		time (s) & V at 22.5$^\circ$C (mL) & V at 50$^\circ$C (mL) \\
		\hline
		30 & 5 & 19 \\
		60 & 8 & 36 \\
		90 & 13 & 48 \\
		120 & 19 & 53 \\
		150 & 24 & 55 \\
		180 & 29 & 57 \\
		210 & 33 & 58 \\
		240 & 37 & 58.2 \\
		270 & 40 & 58.5 \\
		300 & 42 & 59 \\
		330 & 45 & 59.1 \\
		360 & 47 & 59.2 \\
		390 & 49 & 59.3 \\
		420 & 50 & 59.4 \\
		450 & 51 & 59.5 \\
		480 & 52 & 59.9 \\
		510 & 53 & 60 \\
		540 & 53.5 & 60 \\
		570 & 54 & 60.1 \\
		600 & 54.5 & 60.1 \\
		630 & 55 & 60.2 \\
		660 & 55.5 & 60.2 \\
		690 & 56 & 60.3 \\
		720 & 56.1 & 60.5 \\
		750 & 56.3 & 60.6 \\
		780 & 56.5 & 60.8 \\
		810 & 56.9 & 60.9 \\
		840 & 57 & 60.9 \\
		870 & 57.1 & 61 \\
		900 & 57.25 & 61 \\
		930 & 57.5 & 61 \\
		960 & 57.6 & 61 \\
		990 & 57.8 & 61.1
	\end{tabular}
	\end{flushleft}

	\section{Analysis}

	\section{Conclusions}
	
\end{document}
